\documentclass{scrartcl}
\usepackage[margin=3cm]{geometry}
\usepackage{amsmath}
\usepackage{amssymb}
\usepackage{amsthm}
\usepackage{blindtext}
\usepackage{datetime}
\usepackage{fontspec}
\usepackage{graphicx}
\usepackage{kotex}
\usepackage[lighttt]{lmodern}
\usepackage{listings}
\usepackage{mathrsfs}
\usepackage{mathtools}
\usepackage{pgf,tikz,pgfplots}
\usepackage{float}

\pgfplotsset{compat=1.15}
\usetikzlibrary{arrows}

\lstset{
  numbers=none, frame=single, showspaces=false,
  showstringspaces=false, showtabs=false, breaklines=true, showlines=true,
  breakatwhitespace=true, basicstyle=\ttfamily, keywordstyle=\bfseries,
  basewidth=0.5em
}

\newcommand\Overline[2][0.8pt]{%
  \begin{tikzpicture}[baseline=(a.base)]
    \node[inner xsep=0pt,inner ysep=1.5pt] (a) {$#2$};
    \draw[line width= #1] (a.north west) -- (a.north east);
  \end{tikzpicture}
}
\newtheorem{theorem}{Theorem}

\setmainhangulfont{Noto Serif CJK KR}[
  UprightFont=* Light, BoldFont=* Bold,
  Script=Hangul, Language=Korean, AutoFakeSlant,
]
\setsanshangulfont{Noto Sans CJK KR}[
  UprightFont=* DemiLight, BoldFont=* Medium,
  Script=Hangul, Language=Korean
]
\setmathhangulfont{Noto Sans CJK KR}[
  SizeFeatures={
    {Size=-6,  Font=* Medium},
    {Size=6-9, Font=*},
    {Size=9-,  Font=* DemiLight},
  },
  Script=Hangul, Language=Korean
]
\title{MATH230: Homework 14 (due Dec. 11)}
\author{손량(20220323)}
\date{Last compiled on: \today, \currenttime}

\newcommand{\un}[1]{\ensuremath{\ \mathrm{#1}}}
\newcommand{\imag}{\operatorname{Im}}
\newcommand{\real}{\operatorname{Re}}
\newcommand{\Log}{\operatorname{Log}}
\newcommand{\Arg}{\operatorname{Arg}}
\DeclareMathOperator*{\Res}{Res}

\begin{document}
\maketitle

\section{Chapter 10 \#28}
Let \(n_1 := n_2 := 25, \alpha := 0.05, \bar{x}_1 := 20, \bar{x}_2 := 12, s_1
:= 1.5, s_2 := 1.25\). Let \(\mu_1\) and \(\mu_2\) be mean percent absorbency
of cotton fiber and acetate, respectively. Set the null hypothesis \(H_0\) as
\(\mu_1 - \mu_2 = 0\), and the alternative hypothesis \(H_1\) as \(\mu_1 - \mu_2
> 0\). Let \(s_p\) be the square root of the value of pooled sample variance.
Then we can write
\begin{align*}
  \frac{\bar{x}_1 - \bar{x}_2}{s_p \sqrt{n^{-1}_1 + n^{-1}_2}}
  = 20.4859
  > 1.677224
  = t_{n_1 + n_2 - 2, \alpha}
\end{align*}
so we can reject \(H_0\) since there is a strong evidence that \(\mu_1 - \mu_2
> 0\) holds, and the answer for the question is yes.

\section{Chapter 10 \#30}
We can write
\begin{align*}
  z
  := \frac{\bar{x}_1 - \bar{x}_2}
    {\sqrt{\frac{\sigma^2_1}{n_1} + \frac{\sigma^2_2}{n_2}}}
  = 3.66228
\end{align*}
Then, we can write the \(p\)-value as
\begin{align*}
  P(|Z| > z)
  = 2P(Z > z)
  = 0.0002499803
\end{align*}
Since \(p\)-value is sufficiently small, we can reject the null hypothesis.

\section{Chapter 10 \#40}
Using this R code,
\begin{lstlisting}[language=R]
x1 <- c(0.97, 1.16, 0.72, 0.86, 1.00, 0.85, 0.81, 0.58, 0.62, 0.57, 1.32, 0.64, 1.24, 0.98, 0.99, 1.09, 0.90, 0.92, 0.74, 0.78, 0.88, 1.24, 0.94, 1.18)
x2 <- c(0.48, 0.71, 0.98, 0.68, 1.18, 1.36, 0.78, 1.64)
t.test(x1, x2, alternative="two.sided")
\end{lstlisting}
We can conclude that the \(p\)-value is 0.6876, which is not sufficient to
reject the null hypothesis. In other words, the evidence is not sufficient.

\section{Chapter 10 \#44}
Using this R code,
\begin{lstlisting}[language=R]
x1 <- c(224, 270, 400, 444, 590, 660, 1400, 680)
x2 <- c(116, 96, 239, 329, 437, 597, 689, 576)
t.test(x1, x2, alternative="two.sided", paired=TRUE)
\end{lstlisting}
The \(p\)-value is 0.03186, which is sufficient to reject the null hypothesis.
In other words, we can say that the length of storage influences the
concentration in question.

\section{Chapter 10 \#62}
Let \(n := 48, x = 16, p_0 := 1/4, q_0 := 1 - p_0, \alpha := 0.05\), and \(p\)
be the proportion of rats developing tumors. Set the null hypothesis \(H_0\) as
\(p = p_0\), and the alternative hypothesis \(H_1\) as \(p > p_0\). Then we can
write
\begin{align*}
  z
  = \frac{x - np_0}{\sqrt{np_0 q_0}}
  = 1.333333
  < 1.644854
  = z_\alpha
\end{align*}
so we fail to reject \(H_0\).

\section{Chapter 10 \#71}
Let \(n := 25, \sigma^2_0 := 1.15, s^2 := 2.03, \alpha := 0.05\), and
\(\sigma^2\) be the variance of the contents. Set the null hypothesis \(H_0\)
as \(\sigma^2 = \sigma^2_0\), and the alternative hypothesis \(H_1\) as
\(\sigma^2 > \sigma^2_0\). Then we can write
\begin{align*}
  v
  := \frac{(n - 1) s^2}{\sigma^2_0}
  = 42.36522
  > 36.41503
  = \chi^2_{n - 1, \alpha}
\end{align*}
so we can reject \(H_0\).


\section{Chapter 10 \#80}
Using this R code,
\begin{lstlisting}[language=R]
x <- c(14, 18, 32, 20, 16)
chisq.test(x, p=rep(1/5, 5))
\end{lstlisting}
The \(p\)-value is 0.04043, which is sufficient to reject the null hypothesis.
Thus, the distribution of grades is not uniform.

\section{Chapter 10 \#86}
Using this R code,
\begin{lstlisting}[language=R]
m <- matrix(c(21, 36, 30, 48, 26, 19), nrow=2, byrow=TRUE)
chisq.test(m)
\end{lstlisting}
The \(p\)-value is 0.0007232, which is sufficient to reject the null
hypothesis. Thus, we know that hypertension is not independent of smoking
habits.

\section{Chapter 10 \#92}
Using this R code,
\begin{lstlisting}[language=R]
m <- matrix(c(11, 13, 9, 32, 28, 27, 7, 9, 14), nrow=3, byrow=TRUE)
chisq.test(m)
\end{lstlisting}
The \(p\)-value is 0.4323, which is not sufficient to reject the null
hypothesis. Thus, we fail to reject the null hypothesis and know that the
remedies are equally effective.

\end{document}
% vim: textwidth=79
