\documentclass{scrartcl}
\usepackage[margin=3cm]{geometry}
\usepackage{amsmath}
\usepackage{amssymb}
\usepackage{amsthm}
\usepackage{blindtext}
\usepackage{datetime}
\usepackage{fontspec}
\usepackage{graphicx}
\usepackage{kotex}
\usepackage{mathrsfs}
\usepackage{mathtools}
\usepackage{pgf,tikz,pgfplots}
\usepackage{float}

\pgfplotsset{compat=1.15}
\usetikzlibrary{arrows}

\newcommand\Overline[2][0.8pt]{%
  \begin{tikzpicture}[baseline=(a.base)]
    \node[inner xsep=0pt,inner ysep=1.5pt] (a) {$#2$};
    \draw[line width= #1] (a.north west) -- (a.north east);
  \end{tikzpicture}
}
\newtheorem{theorem}{Theorem}

\setmainhangulfont{Noto Serif CJK KR}[
  UprightFont=* Light, BoldFont=* Bold,
  Script=Hangul, Language=Korean, AutoFakeSlant,
]
\setsanshangulfont{Noto Sans CJK KR}[
  UprightFont=* DemiLight, BoldFont=* Medium,
  Script=Hangul, Language=Korean
]
\setmathhangulfont{Noto Sans CJK KR}[
  SizeFeatures={
    {Size=-6,  Font=* Medium},
    {Size=6-9, Font=*},
    {Size=9-,  Font=* DemiLight},
  },
  Script=Hangul, Language=Korean
]
\title{MATH230: Homework 3 (due Sep. 25)}
\author{손량(20220323)}
\date{Last compiled on: \today, \currenttime}

\newcommand{\un}[1]{\ensuremath{\ \mathrm{#1}}}
\newcommand{\imag}{\operatorname{Im}}
\newcommand{\real}{\operatorname{Re}}
\newcommand{\Log}{\operatorname{Log}}
\newcommand{\Arg}{\operatorname{Arg}}
\DeclareMathOperator*{\Res}{Res}

\begin{document}
\maketitle

\section{Chapter 4 \#17}
Let \(S_x = \{-3, 6, 9\}\). Using the definition of expected value, we can
write
\begin{align*}
  \mu_{g(X)} = E[g(X)] = \sum_{x \in S_x} g(x) f(x)
  = \sum_{x \in S_x} (2x + 1)^2 f(x) = 209
\end{align*}

\section{Chapter 4 \#23}
\subsection{Solution for (a)}
Let \(S_x = \{2, 4\}\) and \(S_y = \{1, 3, 5\}\). Using the definition of
expected value, we can write
\begin{align*}
  E[g(X, Y)] = \sum_{x \in S_x} \sum_{y \in S_y} g(x, y) f(x, y)
  = \sum_{x \in S_x} \sum_{y \in S_y} xy^2 f(x, y) = \frac{317}{10}
\end{align*}

\subsection{Solution for (b)}
Using the definition of expected value, we can write
\begin{align*}
  \mu_X &= E(X) = \sum_{x \in S_x} \sum_{y \in S_y} x f(x, y)
  = \frac{29}{10} \\
  \mu_Y &= E(Y) = \sum_{x \in S_x} \sum_{y \in S_y} y f(x, y) = 3
\end{align*}

\section{Chapter 4 \#26}
Using the definition of expected value, we can write
\begin{align*}
  E(Z)
  &= \int^\infty_{-\infty} \int^\infty_{-\infty} \sqrt{x^2 + y^2} f(x, y) dx dy
  = \int^1_0 \int^1_0 \sqrt{x^2 + y^2} \cdot 4xy\; dx dy \\
  &= \int^1_0 \left[ \frac{4}{3} y (x^2 + y^2)^{3/2} \right]^1_0 dy
  = \frac{4}{3} \int^1_0 \left[ y(1 + y^2)^{3 / 2} - y^4 \right] dy \\
  &= \frac{4}{3} \int^1_0 y(1 + y^2)^{3 / 2} dy - \frac{4}{15}
\end{align*}
Using the substituion of \(y = \tan u\), we can write
\begin{align*}
  E(Z)
  &= \frac{4}{3}
  \int^\frac{\pi}{4}_0 (\tan u) (1 + \tan^2 u)^{3 / 2} (\sec^2 u) du
  - \frac{4}{15}
  = \frac{4}{3} \int^\frac{\pi}{4}_0 \tan u \sec^5 u\; du
  - \frac{4}{15} \\
  &= \frac{4}{3} \left[ \frac{1}{5} \sec^5 u \right]^\frac{\pi}{4}_0
  - \frac{4}{15}
  = \frac{16\sqrt{2} - 8}{15}
\end{align*}

\section{Chapter 4 \#43}
Using the definition and property of expected value, we can write
\begin{align*}
  \mu_Y &= E(Y) = E(3X - 2) = 3E(X) - 2 = 3\int^\infty_{-\infty} xf(x) dx - 2 \\
  &= 3 \int^\infty_0 x \cdot \frac{1}{4}e^{-x / 4} dx - 2
  = 3 \left( \lim_{t \to \infty} \left[ (-x - 4) e^{-x / 4} \right]^t_0
    \right) - 2 = 10
\end{align*}
As we can show that \(\lim_{t \to \infty} te^{-t} = 0\) using L'Hôpital's rule.
Also, by the definition and property of variance, we can write
\begin{align*}
  \sigma_Y^2
  &= \sigma_{3X - 2}^2 = 9\sigma_X^2 = 9[E(X^2) - (E(X))^2]
  = 9 \left[ \int^\infty_{-\infty} x^2 f(x) dx
    - \left( \int^\infty_{-\infty} x f(x) dx \right)^2 \right] \\
  &= 9 \left[ \int^\infty_0 x^2 \cdot \frac{1}{4} e^{-x / 4} dx
    - \left( \int^\infty_0 x \cdot \frac{1}{4} e^{-x / 4} dx \right)^2
    \right] \\
  &= 9 \left[ \lim_{t \to \infty} \left[ (-x^2 - 8x - 32) e^{-x / 4} \right]^t_0
    - \left( \lim_{t \to \infty}
    \left[ (-x - 4)e^{-x / 4} \right]^t_0 \right)^2 \right] \\
  &= 9 (32 - 16) = 144
\end{align*}

\section{Chapter 4 \#52}
First, we can write
\begin{align*}
  \mu_X
  &= E(X) = \int^\infty_{-\infty} \int^\infty_{-\infty} x f(x, y) dx dy
  = \int^1_0 \int^y_0 x f(x, y) dx dy \\
  &= \int^1_0 \left[ x^2 \right]^y_0 dy = \int^1_0 y^2\; dy = \frac{1}{3} \\
  \mu_Y
  &= E(Y) = \int^\infty_{-\infty} \int^\infty_{-\infty} y f(x, y) dx dy
  = \int^1_0 \int^y_0 2y\; dx dy \\
  &= \int^1_0 2y^2\; dy = \left[ \frac{2}{3}y^3 \right]^1_0 = \frac{2}{3}
\end{align*}
Using the definition of variance, we can write
\begin{align*}
  \sigma^2_X
  &= E(X^2) - \mu_X^2
  = \int^\infty_{-\infty} \int^\infty_{-\infty} x^2 f(x, y) dx dy - \frac{1}{9}
  = \int^1_0 \int^y_0 2x^2 dx dy - \frac{1}{9} \\
  &= \int^1_0 \left[ \frac{2}{3}x^3 \right]^y_0 dy - \frac{1}{9}
  = \int^1_0 \frac{2}{3}y^3 dy - \frac{1}{9} = \frac{1}{18} \\
  \sigma^2_Y
  &= E(Y^2) - \mu_Y^2
  = \int^\infty_{-\infty} \int^\infty_{-\infty} y^2 f(x, y) dx dy - \frac{4}{9}
  = \int^1_0 \int^y_0 2y^2 dx dy - \frac{4}{9} \\
  &= \int^1_0 \left[ 2y^2 x \right]^y_0 dy - \frac{4}{9}
  = \int^1_0 2y^3 dy - \frac{4}{9} = \frac{1}{18}
\end{align*}
Using the definition of covariance, we can write
\begin{align*}
  \sigma_{XY}
  &= E(XY) - \mu_X \mu_Y
  = \int^\infty_{-\infty} \int^\infty_{-\infty}
    xy f(x, y) dx dy - \frac{2}{9} \\
  &= \int^1_0 \int^y_0 2xy\; dx dy - \frac{2}{9}
  = \int^1_0 \left[ x^2 y \right]^y_0 dy - \frac{2}{9}
  = \int^1_0 y^3\; dy - \frac{2}{9} = \frac{1}{36}
\end{align*}
By the definition of correlation,
\begin{align*}
  \rho_{XY} = \frac{\sigma_{XY}}{\sigma_X \sigma_Y} = \frac{1}{2}
\end{align*}

\section{Chapter 4 \#70}
\subsection{Solution for (a)}
Considering the marginal densities \(g(x)\) and \(h(y)\) for \(0 \leq x \leq
1\) and \(0 \leq y \leq 1\),
\begin{align*}
  g(x)
  &= \int^\infty_{-\infty} f(x, y) dy
  = \frac{3}{2} \int^1_0 (x^2 + y^2) dy
  = \frac{3}{2} \left[ x^2 y + \frac{1}{3} y^3 \right]^1_0
  = \frac{3}{2} \left( x^2 + \frac{1}{3} \right) \\
  h(y)
  &= \int^\infty_{-\infty} f(x, y) dx
  = \frac{3}{2} \int^1_0 (x^2 + y^2) dx
  = \frac{3}{2} \left[ \frac{1}{3}x^3 + xy^2 \right]^1_0
  = \frac{3}{2} \left( y^2 + \frac{1}{3} \right)
\end{align*}
Take \(x = 1\) and \(y = 1\), then \(f(x, y) = 3 \not = g(1) h(1) = 4\). Thus,
\(X\) and \(Y\) are dependent.

\subsection{Solution for (b)}
By property of expected value, we can write
\begin{align*}
  E(X + Y)
  &= E(X) + E(Y)
  = \int^\infty_{-\infty} x g(x) dx + \int^\infty_{-\infty} y h(y) dy
  = \int^1_0 x g(x) dx + \int^1_0 y h(y) dy \\
  &= \int^1_0 \frac{3}{2} x \left( x^2 + \frac{1}{3} \right) dx
  + \int^1_0 \frac{3}{2} y \left( y^2 + \frac{1}{3} \right) dy
  = \left[ \frac{3}{8} x^4 + \frac{1}{4} x^2 \right]^1_0
  + \left[ \frac{3}{8} y^4 + \frac{1}{4} y^2 \right]^1_0 \\
  &= \frac{5}{4}
\end{align*}
Also, by the definition of expected value,
\begin{align*}
  E(XY)
  &= \int^\infty_{-\infty} \int^\infty_{-\infty} xy f(x, y) dx dy
  = \int^1_0 \int^1_0 \frac{3}{2} xy(x^2 + y^2) dx dy \\
  &= \int^1_0 \left[ \frac{3}{8} x^4 y + \frac{3}{4} x^2 y^3 \right]^1_{x = 0} dy
  = \int^1_0 \left( \frac{3}{8} y + \frac{3}{4} y^3 \right) dy \\
  &= \left[ \frac{3}{16} y^2 + \frac{3}{16} y^4 \right]^1_0
  = \frac{3}{8}
\end{align*}

\subsection{Solution for (c)}
By a property of variance, we can write
\begin{align*}
  \mathrm{Var}(X)
  &= E(X^2) - \mu^2_X
  = \int^\infty_{-\infty} x^2 g(x) dx - \mu^2_X
  = \int^1_0 \frac{3}{2} x^2 \left( x^2 + \frac{1}{3} \right) dx
    - \left( \frac{5}{8} \right)^2 \\
  &= \left[ \frac{3}{10} x^5 + \frac{1}{6} x^3 \right]^1_0
    - \left( \frac{5}{8} \right)^2
  = \frac{73}{960} \\
  \mathrm{Var}(Y)
  &= E(Y^2) - \mu^2_Y
  = \int^\infty_{-\infty} y^2 h(y) dy - \mu^2_Y
  = \int^1_0 \frac{3}{2} y^2 \left( y^2 + \frac{1}{3} \right) dy
    - \left( \frac{5}{8} \right)^2 \\
  &= \left[ \frac{3}{10} y^5 + \frac{1}{6} y^3 \right]^1_0
    - \left( \frac{5}{8} \right)^2
  = \frac{73}{960}
\end{align*}
By a property of covariance,
\begin{align*}
  \mathrm{Cov}(X, Y)
  = E(XY) - \mu_X \mu_Y
  = \frac{3}{8} - \frac{5}{8} \cdot \frac{5}{8} = -\frac{1}{64}
\end{align*}

\subsection{Solution for (d)}
By properties of variance and covariance,
\begin{align*}
  \mathrm{Var}(X + Y)
  = \mathrm{Var}(X) + \mathrm{Var}(Y) + 2\mathrm{Cov}(X, Y)
  = \frac{73}{960} + \frac{73}{960} + 2 \cdot \left( -\frac{1}{64} \right)
  = \frac{29}{240}
\end{align*}

\section{Chapter 4 \#78}
By the definition of expected value, we can write
\begin{align*}
  \mu
  &= E(X)
  = \int^\infty_{-\infty} xf(x) dx
  = \int^1_0 30x^3 (1 - x)^2 dx
  = 30 \int^1_0 (x^3 - 2x^4 + x^5) dx \\
  &= 30 \left[ \frac{1}{4} x^4 - \frac{2}{5} x^5 + \frac{1}{6} x^6 \right]^1_0
  = \frac{1}{2}
\end{align*}
By the definition of variance,
\begin{align*}
  \sigma^2
  &= E(X^2) - \mu^2
  = \int^\infty_{-\infty} x^2 f(x) dx - \mu^2
  = \int^1_0 30x^4 (1 - x)^2 dx - \mu^2 \\
  &= 30 \int^1_0 (x^4 - 2x^5 + x^6) dx - \mu^2
  = 30 \left[ \frac{1}{5} x^5 - \frac{1}{3} x^6 + \frac{1}{7} x^7 \right]^1_0
    - \mu^2
  = \frac{2}{7} - \frac{1}{4}
  = \frac{1}{28}
\end{align*}
Thus, we can write
\begin{align*}
  P(\mu - 2\sigma < X < \mu + 2\sigma)
  &= \int^{\mu + 2\sigma}_{\mu - 2\sigma} f(x) dx
  = \int^{\frac{1}{2} + \frac{1}{\sqrt{7}}}_{\frac{1}{2} - \frac{1}{\sqrt{7}}}
    30x^2(1 - x)^2 dx \\
  &= 30 \left[ \frac{1}{3} x^3 - \frac{1}{2} x^4 + \frac{1}{5} x^5 \right]^{\frac{1}{2} + \frac{1}{\sqrt{7}}}_{\frac{1}{2} - \frac{1}{\sqrt{7}}}
  = 0.96998
\end{align*}
The lower bound given by Chebyshev's theorem is \(1 - 2^{-2} = 3/4\).

\section{Lecture Note Exercise \#4.1}
\subsection{Solution for (a)}
Let \(S_x = \{0, 1, 2\}\) and \(S_y = \{1, 2\}\). We can write
\begin{align*}
  \mu_X
  &= \sum_{x \in S_x} xg(x) = 1 \\
  \sigma^2_X
  &= E(X^2) - \mu^2_X
  = \sum_{x \in S_x} x^2 g(x) - \mu^2_X
  = \frac{14}{9} - 1 = \frac{5}{9}
\end{align*}

\subsection{Solution for (b)}
Let \(m(y) = \sum_{x \in S_x} x f(x | y)\). Then we can write
\begin{align*}
  m(1)
  &= \sum_{x \in S_x} x f(x | y = 1)
  = \sum_{x \in S_x} x \frac{f(x, 1)}{h(1)}
  = \frac{f(1, 1)}{h(1)} + \frac{2f(2, 1)}{h(1)} \\
  &= \frac{4}{18} \cdot \left( \frac{1}{2} \right)^{-1}
    + 2 \left( \frac{1}{18} \right) \left( \frac{1}{2} \right)^{-1}
  = \frac{2}{3} \\
  m(2)
  &= \sum_{x \in S_x} x f(x | y = 1)
  = \sum_{x \in S_x} x \frac{f(x, 2)}{h(2)}
  = \frac{f(1, 2)}{h(2)} + \frac{2f(2, 2)}{h(2)} \\
  &= \frac{4}{18} \cdot \left( \frac{1}{2} \right)^{-1}
    + 2 \left( \frac{4}{18} \right) \left( \frac{1}{2} \right)^{-1}
  = \frac{4}{3}
\end{align*}
We can write
\begin{align*}
  \mu_X
  = E[m(Y)]
  = \sum_{y \in S_y} m(y) h(y)
  = m(1) h(1) + m(2) h(2)
  = \frac{2}{3} \cdot \frac{1}{2} + \frac{4}{3} \cdot \frac{1}{2}
  = 1
\end{align*}
This conicides with the result of (a).

\subsection{Solution for (c)}
Let \(v(y) = E(X^2 | y) - (m(y))^2\). Then
\begin{align*}
  v(1)
  &= E(X^2 | y = 1) - (m(1))^2
  = \sum_{x \in S_x} x^2 f(x | y = 1) - (m(1))^2 \\
  &= \frac{f(1, 1)}{h(1)} + \frac{4f(2, 1)}{h(1)} - (m(1))^2
  = \frac{4}{18} \cdot \left( \frac{1}{2} \right)^{-1}
    + 4 \left( \frac{1}{18} \right) \left( \frac{1}{2} \right)^{-1}
    - \left( \frac{2}{3} \right)^{-1}
  = \frac{4}{9} \\
  v(2)
  &= E(X^2 | y = 2) - (m(2))^2
  = \sum_{x \in S_x} x^2 f(x | y = 2) - (m(2))^2 \\
  &= \frac{f(1, 2)}{h(2)} + \frac{4f(2, 2)}{h(2)} - (m(2))^2
  = \frac{4}{18} \cdot \left( \frac{1}{2} \right)^{-1}
    + 4 \left( \frac{4}{18} \right) \left( \frac{1}{2} \right)^{-1}
    - \left( \frac{4}{3} \right)^{-1}
  = \frac{4}{9}
\end{align*}
We can write
\begin{align*}
  \sigma^2_X
  &= E[v(Y)] + \mathrm{Var}[m(Y)]
  = \sum_{y \in S_y} v(y) h(y) + \sum_{y \in S_y} (m(y))^2 h(y)
    - \left( \sum_{y \in S_y} m(y) h(y) \right)^2 \\
  &= v(1) h(1) + v(2) h(2) + (m(1))^2 h(1) + (m(2))^2 h(2)
    - (m(1) h(1) + m(2) h(2))^2
  = \frac{5}{9}
\end{align*}
As \(m(y)\) is the expected value of \(X\) when given a value of \(Y\), the
variation from randomness in coin selection is captured by
\(\mathrm{Var}[m(Y)]\). The term can be calculated as
\begin{align*}
  \mathrm{Var}[m(Y)]
  &= \sum_{y \in S_y} (m(y))^2 h(y)
    - \left( \sum_{y \in S_y} m(y) h(y) \right)^2 \\
  &= (m(1))^2 h(1) + (m(2))^2 h(2) - (m(1) h(1) + m(2) h(2))^2
  = \frac{1}{9}
\end{align*}

\end{document}
% vim: textwidth=79
