\documentclass{scrartcl}
\usepackage[margin=3cm]{geometry}
\usepackage{amsmath}
\usepackage{amssymb}
\usepackage{amsthm}
\usepackage{blindtext}
\usepackage{datetime}
\usepackage{fontspec}
\usepackage{graphicx}
\usepackage{kotex}
\usepackage{mathrsfs}
\usepackage{mathtools}
\usepackage{pgf,tikz,pgfplots}
\usepackage{float}

\pgfplotsset{compat=1.15}
\usetikzlibrary{arrows}

\newcommand\Overline[2][0.8pt]{%
  \begin{tikzpicture}[baseline=(a.base)]
    \node[inner xsep=0pt,inner ysep=1.5pt] (a) {$#2$};
    \draw[line width= #1] (a.north west) -- (a.north east);
  \end{tikzpicture}
}
\newtheorem{theorem}{Theorem}

\setmainhangulfont{Noto Serif CJK KR}[
  UprightFont=* Light, BoldFont=* Bold,
  Script=Hangul, Language=Korean, AutoFakeSlant,
]
\setsanshangulfont{Noto Sans CJK KR}[
  UprightFont=* DemiLight, BoldFont=* Medium,
  Script=Hangul, Language=Korean
]
\setmathhangulfont{Noto Sans CJK KR}[
  SizeFeatures={
    {Size=-6,  Font=* Medium},
    {Size=6-9, Font=*},
    {Size=9-,  Font=* DemiLight},
  },
  Script=Hangul, Language=Korean
]
\title{MATH230: Homework 12 (due Nov. 27)}
\author{손량(20220323)}
\date{Last compiled on: \today, \currenttime}

\newcommand{\un}[1]{\ensuremath{\ \mathrm{#1}}}
\newcommand{\imag}{\operatorname{Im}}
\newcommand{\real}{\operatorname{Re}}
\newcommand{\Log}{\operatorname{Log}}
\newcommand{\Arg}{\operatorname{Arg}}
\DeclareMathOperator*{\Res}{Res}

\begin{document}
\maketitle

\section{Chapter 9 \#43}
Let \(n_1 = n_2 = 12, \bar{x}_1 = 36300, s_1 = 5000 \bar{x}_2 = 38100, s_2 =
6100, \alpha = 0.05\). Using the formula for approximate confidence interval,
we can write
\begin{align*}
  \nu
  &= \frac{(s^2_1 / n_1 + s^2_2 / n_2)^2}
    {(s^2_1 / n_1)^2 / (n_1 - 1) + (s^2_2 / n_2)^2 / (n_2 - 1)} \\
\end{align*}
and
\begin{align*}
  (\bar{x}_1 - \bar{x}_2) \pm t_{\nu, \alpha / 2}
    \sqrt{\frac{s^2_1}{n_1} + \frac{s^2_2}{n_2}}
  = -1800 \pm 4732.522
\end{align*}
This result can be written as \((-6532.522, 2932.522)\).

\section{Chapter 9 \#56}
\subsection{Solution for (a)}
Let \(\hat{p} = 90 / 120, \hat{q} = 1 - \hat{p}, n = 120, \alpha = 0.05\).
Using the formula for approximate confidence interval, we can write
\begin{align*}
  \hat{p} \pm z_{\alpha / 2} \sqrt{\frac{\hat{p} \hat{q}}{n}}
  = 0.75 \pm 0.07747438
\end{align*}
The result can be written as \((0.6725256, 0.8274744)\).

\subsection{Solution for (b)}
If \(\hat{p}\) is used as an estimate of the proportion, we are \(100(1 -
\alpha)\%\) confident that the error will not exceed the following value:
\begin{align*}
  z_{\alpha / 2} \sqrt{\frac{\hat{p} \hat{q}}{n}}
  = 0.07747438
\end{align*}

\section{Chapter 9 \#67}
\subsection{Solution for (a)}
Let \(\hat{p} = 2 / 3, \hat{q} = 1 - \hat{p}, n = 1600, \alpha = 0.05\). Using
the formula for approximate confidence interval, we can write
\begin{align*}
  \hat{p} \pm z_{\alpha / 2} \sqrt{\frac{\hat{p} \hat{q}}{n}}
  = 0.6666667 \pm 0.0230984
\end{align*}
The result can be written as \((0.6435683, 0.6897651)\).

\subsection{Solution for (b)}
If \(\hat{p}\) is used as an estimate of the proportion, we are \(100(1 -
\alpha)\%\) confident that the error will not exceed the following value:
\begin{align*}
  z_{\alpha / 2} \sqrt{\frac{\hat{p} \hat{q}}{n}}
  = 0.0230984
\end{align*}

\section{Chapter 9 \#72}
Let \(n = 24, s^2 = 14, \alpha = 0.05\). Using the formula for confidence
interval, we can write
\begin{align*}
  \left( \frac{(n - 1) s^2}{\chi^2_{(n - 1), \alpha / 2}},
    \frac{(n - 1) s^2}{\chi^2_{(n - 1), 1 - \alpha / 2}} \right)
  = (8.456853, 27.54832)
\end{align*}

\section{Chapter 9 \#78}
Let \(n_1 = 5, n_2 = 7, \alpha = 0.1\). We can calculate sample mean and sample
variance as
\begin{align*}
  \bar{x}_1
  &= \frac{103 + 94 + 110 + 87 + 98}{5}
  = 98.4 \\
  \bar{x}_2
  &= \frac{97 + 82 + 123 + 92 + 175 + 88 + 118}{7}
  = 110.7143 \\
  s_1^2
  &= \frac{(103 - \bar{x}_1)^2 + \dots + (98 - \bar{x}_1)^2}{5 - 1}
  = 76.3 \\
  s_2^2
  &= \frac{(97 - \bar{x}_2)^2 + \dots + (118 - \bar{x}_2)^2}{7 - 1}
  = 1035.905
\end{align*}
Using the formula for confidence interval, we can write
\begin{align*}
  \left( \frac{1}{f_{(n_1 - 1, n_2 - 1), \alpha / 2}}
    \frac{s^2_1}{s^2_2},
    \frac{1}{f_{(n_1 - 1, n_2 - 1), 1 - \alpha / 2}}
    \frac{s^2_1}{s^2_2} \right)
  = (0.01624629, 0.4539481)
\end{align*}
Since the confidence interval doesn't contain 1, it is reasonable to assume
that \(\sigma^2_1 \not = \sigma^2_2\).

\end{document}
% vim: textwidth=79
