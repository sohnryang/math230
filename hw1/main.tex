\documentclass{scrartcl}
\usepackage[margin=3cm]{geometry}
\usepackage{amsmath}
\usepackage{amssymb}
\usepackage{amsthm}
\usepackage{datetime}
\usepackage{fontspec}
\usepackage{graphicx}
\usepackage{kotex}
\usepackage{mathrsfs}
\usepackage{mathtools}
\usepackage{pgf,tikz,pgfplots}

\pgfplotsset{compat=1.15}
\usetikzlibrary{arrows}

\newcommand\Overline[2][0.8pt]{%
  \begin{tikzpicture}[baseline=(a.base)]
    \node[inner xsep=0pt,inner ysep=1.5pt] (a) {$#2$};
    \draw[line width= #1] (a.north west) -- (a.north east);
  \end{tikzpicture}
}
\newtheorem{theorem}{Theorem}

\setmainhangulfont{Noto Serif CJK KR}[
UprightFont=* Light, BoldFont=* Bold,
Script=Hangul, Language=Korean, AutoFakeSlant,
]
\setsanshangulfont{Noto Sans CJK KR}[
UprightFont=* DemiLight, BoldFont=* Medium,
Script=Hangul, Language=Korean
]
\setmathhangulfont{Noto Sans CJK KR}[
SizeFeatures={
  {Size=-6,  Font=* Medium},
  {Size=6-9, Font=*},
  {Size=9-,  Font=* DemiLight},
},
Script=Hangul, Language=Korean
]
\title{MATH230: Homework 1 (due Sep. 11)}
\author{손량(20220323)}
\date{Last compiled on: \today, \currenttime}

\newcommand{\un}[1]{\ensuremath{\ \mathrm{#1}}}
\newcommand{\imag}{\operatorname{Im}}
\newcommand{\real}{\operatorname{Re}}
\newcommand{\Log}{\operatorname{Log}}
\newcommand{\Arg}{\operatorname{Arg}}
\DeclareMathOperator*{\Res}{Res}

\begin{document}
\maketitle

\section{Chapter 2 \#12}

\subsection{Solution for (a)}
The subjects' type can be encoded using a string of length 3, \(A_1 A_2 A_3\)
where \(A_1 \in \{Z, W, S\}, A_2 \in \{Y, N\}, A_3 \in \{M, F\}\). Then the
sample space can be written as follows:
\begin{align*}
  S = \{ZYM, ZYF, ZNM, ZNF, WYM, WYF, WNM, WNF, SYM, SYF, SNM, SNF\}
\end{align*}

\subsection{Solution for (b)}
\(A\) can we written as:
\begin{align*}
  A = \{ZYF, ZNF, WYF, WNF, SYF, SNF\}
\end{align*}
\(B\) can be written as:
\begin{align*}
  B = \{WYM, WYF, WNM, WNF\}
\end{align*}
So we can write
\begin{align*}
  A \cup B = \{ZYF, ZNF, WYM, WYF, WNM, WNF, SYF, SNF\}
\end{align*}

\subsection{Solution for (c)}
Using \(A\) and \(B\) listed above, we can write
\begin{align*}
  A \cap B = \{WYF, WNF\}
\end{align*}

\section{Chapter 2 \#32}
\subsection{Solution for (a)}
The number of ways is equal to the number of permutations of 6 objects, which
is \(6! = 720\).

\subsection{Solution for (b)}
First, there are \(3! = 6\) ways to line up the three stubborn people. Then,
treating those people as one person, there are \(4! = 24\) ways to line up 3
other people and the three-person-group. Using the rule~2.2 in the textbook,
we know that there are \(3! \times 4! = 144\) ways to line up.

\subsection{Solution for (c)}
First, there are \(4! = 24\) ways to line up 4 people, except the two people
who refuse to follow each other. Then, there are \(5 \choose 2\) positions
between, front, or rear of the already lined-up people. Also, There are \(2!\)
ways to determine one of the two people who comes first in line. Using the
rule~2.2 in the textbook, we know that there are
\(4! \times {5 \choose 2} \times 2! = 480\) ways to line up.

\section{Chapter 2 \#38}
\subsection{Solution for (a)}
The number of ways is equal to the number of permutations of 6 objects, which
is \(6! = 720\).

\subsection{Solution for (b)}
There are \(3! = 6\) ways to seat three couples, and each couple has \(2!\)
ways to sit: male sits left or female sits left in row. By the rule~2.2 in the
textbook, the number of ways is \(3! \times (2!)^3 = 48\).

\subsection{Solution for (c)}
There are \(3!\) ways to seat 3 women, and \(3!\) ways to seat 3 men. By the
rule~2.2 in the textbook, the number of ways is \(3! \times 3! = 36\).

\section{Chapter 2 \#43}
By the theorem~2.3, the number of ways is \((6 - 1)! = 120\).

\section{Chapter 2 \#58}
\subsection{Solution for (a)}
Considering two dice distinct, let's denote outcome of the dice throw as
\((a, b)\), where \(a\) and \(b\) are the numbers from each dice. Then, the
sample space can be written as follows:
\begin{align*}
  S = \{(1, 1), (1, 2), \dots, (1, 6), (2, 1), (2, 2), \dots, (2, 6), \dots,
  (6, 1), (6, 2), \dots, (6, 6)\}
\end{align*}
If \(A\) represents the event of the total being 9, we can write
\begin{align*}
  A = \{(3, 6), (4, 5), (5, 4), (6, 3)\}
\end{align*}
Since the dice is fair, each of the outcomes in \(S\) is equally likely to
occur. We assign a probability of \(\omega\) to each sample point of \(S\).
Then \(36\omega = 1\) so \(\omega = 1 / 36\). Thus,
\begin{align*}
  P(A) = 4 \times \frac{1}{36} = \frac{1}{9}
\end{align*}

\subsection{Solution for (b)}
If \(B\) represents the event of the total being at most 3, we can write
\begin{align*}
  B = \{(1, 1), (1, 2), (2, 1)\}
\end{align*}
Using the similar argument we made in (a),
\begin{align*}
  P(B) = 3 \times \frac{1}{36} = \frac{1}{12}
\end{align*}

\section{Chapter 2 \#72}
Using de Morgan's~law, we can write
\[ P(A' \cap B') = P((A \cup B)') \]
Using theorem~2.9 in the textbook,
\[ P((A \cup B)') = 1 - P(A \cup B) \]
By theorem~2.7 in the textbook,
\[ 1 - P(A \cup B) = 1 - (P (A) + P(B) - P(A \cap B)) \]
Then we get the desired result:
\[ P(A' \cap B') = 1 + P(A \cap B) - P (A) - P(B) \]

\section{Chapter 2 \#76}
Consider the events:
\begin{itemize}
  \item \(S_0\): The person is a nonsmoker.
  \item \(S_1\): The person is a moderate smoker.
  \item \(S_2\): The person is a heavy smoker.
  \item \(H\): The person is experiencing hypertension.
  \item \(NH\): The person is not experiencing hypertension.
\end{itemize}

\subsection{Solution for (a)}
The probability we are interested in is \(P(H | S_2)\). Using the
definition~2.10,
\begin{align*}
  P(H | S_2)
  = \frac{P(H \cap S_2)}{P(S_2)}
  = \frac{30}{180} \times \left( \frac{49}{180} \right)^{-1}
  = \frac{30}{49}
\end{align*}

\subsection{Solution for (b)}
The probability we are interested in is \(P(S_0 | NH)\). Using the
definition~2.10,
\begin{align*}
  P(S_0 | NH)
  = \frac{P(S_0 \cap NH)}{P(NH)}
  = \frac{48}{180} \times \left( \frac{48 + 26 + 19}{180} \right)^{-1}
  = \frac{48}{93}
\end{align*}

\section{Chapter 2 \#78}
Consider the events:
\begin{itemize}
  \item \(R_1\): The batch is rejected by the first inspection.
  \item \(R_2\): The batch is rejected by the second inspection.
  \item \(R_3\): The batch is rejected by the third inspection.
\end{itemize}

Before we begin, let's prove this statement:
\begin{theorem}
  Let \(A, B\) be independent events of a sample space \(S\), where
  \(0 < P(A) < 1\) and \(0 < P(B) < 1\). Then \(A'\) and \(B\) are independent.
\end{theorem}
\begin{proof}
  \(A\) and \(B\) are independent, so by the theorem~2.11, we can write
  \begin{align*}
    P(B)
    &= P(S \cap B)
    = P((A \cup A') \cap B)
    = P((A \cap B) \cup (A' \cap B)) \\
    &= P(A \cap B) + P(A' \cap B) - P((A \cap B) \cap (A' \cap B)) \\
    &= P(A \cap B) + P(A' \cap B) - P(\varnothing)
    = P(A)P(B) + P(A')P(B)
  \end{align*}
  Then,
  \begin{align*}
    P(A')P(B) = P(B) - P(A)P(B) = (1 - P(A))P(B) = P(A')P(B)
  \end{align*}
  Again, by the theorem~2.11, \(A'\) and \(B\) are independent.
\end{proof}

\subsection{Solution for (a)}
The probability we are interested in is \(P(R_1' \cap R_2)\). As we proved
earlier, \(R_1'\) and \(R_2\) are independent as \(R_1\) and \(R_2\) are
independent. Then we can write
\begin{align*}
  P(R_1' \cap R_2) = P(R_1')P(R_2) = (1 - 0.10) \times 0.05 = 0.045
\end{align*}

\subsection{Solution for (b)}
The probability we are interested in is \(P(R_1' \cap R_2' \cap R_3)\). As we
proved earlier, \(R_1'\) and \(R_2'\) are independent as \(R_1\) and \(R_2\)
are independent, so \(P(R_2' | R_1') = P(R_2')\). In a similar fashion,
\(R_2'\) and \(R_3\), \(R_1'\) and \(R_3\) are independent. By theorem~2.12, we
can write
\begin{align*}
  P(R_1' \cap R_2' \cap R_3)
  = P(R_1')P(R_2')P(R_3)
  = (1 - 0.10) \times (1 - 0.05) \times 0.15
  = 0.12825
\end{align*}

\section{Chapter 2 \#87}
Consider the events:
\begin{itemize}
  \item \(U\): The home is one of the homes that are left unlocked.
  \item \(K\): The home can be opened by one of the three master keys.
\end{itemize}

The probability we are interested in is \(P(U \cup K)\), and \(U\) and \(K\)
can be considered independent as the key was randomly selected. We can write
\begin{align*}
  P(U \cup K)
  = P(U) + P(K) - P(U)P(K)
  = 0.3 + \frac{{7 \choose 2}}{{8 \choose 3}}
  - 0.3 \times \frac{{7 \choose 2}}{{8 \choose 3}}
  = \frac{9}{16}
\end{align*}

\section{Chapter 2 \#93}
Let \(A, B, C, D, E\) be events where the component \(A, B, C, D, E\) works,
respectively. Then we can write
\begin{align*}
  P(A) = P(B) = 0.7, P(C) = P(D) = P(E) = 0.8
\end{align*}

\subsection{Solution for (a)}
The system works if \(A, B\) works, or \(C, D, E\) works. Thus, the probability
we are interested in is \(P((A \cap B) \cup (C \cap D \cap E))\). Since the
components fail independently,
\begin{align*}
  P((A \cap B) \cup (C \cap D \cap E))
  &= P(A \cap B) + P(C \cap D \cap E) - P(A \cap B \cap C \cap D \cap E) \\
  &= P(A)P(B) + P(C)P(D)P(E) - P(A)P(B)P(C)P(C)P(E) \\
  &= 0.75112
\end{align*}

\subsection{Solution for (b)}
Let \(S\) be an event where the system works. The probability we are interested
in is \(P(A' | S)\). We can write
\begin{align*}
  P(A' | S)
  &= \frac{P(A' \cap S)}{P(S)}
  = \frac{P(A' \cap [(A \cap B) \cup (C \cap D \cap E)])}{P((A \cap B) \cup (C \cap D \cap E))} \\
  &= \frac{P(A' \cap C \cap D \cap E)}{P((A \cap B) \cup (C \cap D \cap E))}
  = \frac{1920}{9389}
\end{align*}

\section{Chapter 2 \#96}
Let \(A_i \quad (i = 1, 2, 3, 4)\) be events where the driver is passing
\(L_i\), respectively. Then,
\begin{align*}
  P(A_1) = 0.2, P(A_2) = 0.1, P(A_3) = 0.5, P(A_4) = 0.2
\end{align*}
Let \(B\) the event where radar traps resulting in speeding ticket. Then, the
probability we are interested in is \(P(B)\). We can write
\begin{align*}
  P(B | A_1) = 0.4, P(B | A_2) = 0.3, P(B | A_3) = 0.2, P(B | A_4) = 0.3
\end{align*}
Using the theorem~2.13, we can write
\begin{align*}
  P(B) = \sum^4_{k = 1} P(B | A_k) P(A_k) = 0.27
\end{align*}

\section{Chapter 2 \#98}
The probability we are interested in is \(P(A_2 | B)\). Using the Bayes'~rule,
\begin{align*}
  P(A_2 | B)
  = \frac{P(A_2 \cap B)}{\sum^4_{k = 1} P(A_k \cap B)}
  = \frac{P(B | A_2) P(A_2)}{\sum^4_{k = 1} P(A_k) P(B | A_k)}
  = \frac{1}{9}
\end{align*}

\section{Chapter 2 \#102}
Let \(H\) be the event where you picked \(A\) and the host opened \(B\)
revealing no prize. Since the door is chosen randomly,
\(P(A) = P(B) = P(C) = 1/3\). Assuming that the host does not open the chosen
door or the door which has the prize, we know that
\begin{align*}
  P(H | A) = \frac{1}{2}, P(H | B) = 0, P(H | C) = 1
\end{align*}
Using the Bayes'~rule, the probability of getting the prize if you don't switch
is
\begin{align*}
  P(A | H)
  &= \frac{P(A \cap H)}{P(A \cap H) + P(B \cap H) + P(C \cap H)} \\
  &= \frac{P(H | A) P(A)}{P(H | A) P(A) + P(H | B) P(B) + P(H | C) P(C)}
  = \frac{1}{3}
\end{align*}
However, if you switch,
\begin{align*}
  P(C | H)
  &= \frac{P(C \cap H)}{P(A \cap H) + P(B \cap H) + P(C \cap H)} \\
  &= \frac{P(H | C) P(C)}{P(H | A) P(A) + P(H | B) P(B) + P(H | C) P(C)}
  = \frac{2}{3}
\end{align*}
Thus, switching door is better.

\end{document}
