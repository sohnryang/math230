\documentclass{scrartcl}
\usepackage[margin=3cm]{geometry}
\usepackage{amsmath}
\usepackage{amssymb}
\usepackage{amsthm}
\usepackage{blindtext}
\usepackage{datetime}
\usepackage{fontspec}
\usepackage{graphicx}
\usepackage{kotex}
\usepackage{mathrsfs}
\usepackage{mathtools}
\usepackage{pgf,tikz,pgfplots}
\usepackage{float}

\pgfplotsset{compat=1.15}
\usetikzlibrary{arrows}

\newcommand\Overline[2][0.8pt]{%
  \begin{tikzpicture}[baseline=(a.base)]
    \node[inner xsep=0pt,inner ysep=1.5pt] (a) {$#2$};
    \draw[line width= #1] (a.north west) -- (a.north east);
  \end{tikzpicture}
}
\newtheorem{theorem}{Theorem}

\setmainhangulfont{Noto Serif CJK KR}[
  UprightFont=* Light, BoldFont=* Bold,
  Script=Hangul, Language=Korean, AutoFakeSlant,
]
\setsanshangulfont{Noto Sans CJK KR}[
  UprightFont=* DemiLight, BoldFont=* Medium,
  Script=Hangul, Language=Korean
]
\setmathhangulfont{Noto Sans CJK KR}[
  SizeFeatures={
    {Size=-6,  Font=* Medium},
    {Size=6-9, Font=*},
    {Size=9-,  Font=* DemiLight},
  },
  Script=Hangul, Language=Korean
]
\title{MATH230: Homework 6 (due Oct. 16)}
\author{손량(20220323)}
\date{Last compiled on: \today, \currenttime}

\newcommand{\un}[1]{\ensuremath{\ \mathrm{#1}}}
\newcommand{\imag}{\operatorname{Im}}
\newcommand{\real}{\operatorname{Re}}
\newcommand{\Log}{\operatorname{Log}}
\newcommand{\Arg}{\operatorname{Arg}}
\DeclareMathOperator*{\Res}{Res}

\begin{document}
\maketitle

\section{Chapter 6 \#1}
\subsection{Solution for (a)}
We can write
\begin{align*}
  \mu
  = \int^\infty_{-\infty} x f(x; A, B) dx
  = \int^B_A \frac{x}{B - A} dx
  = \left[ \frac{x^2}{2(B - A)} \right]^B_A
  = \frac{A + B}{2}
\end{align*}

\subsection{Solution for (b)}
We can write
\begin{align*}
  \sigma^2
  &= \int^\infty_{-\infty} (x - \mu)^2 f(x; A, B) dx
  = \int^B_A \left( x - \frac{A + B}{2} \right)^2 \frac{1}{B - A} dx \\
  &= \frac{1}{B - A}
    \left[ \frac{1}{3} \left( x - \frac{A + B}{2} \right)^3 \right]^B_A
  = \frac{1}{3(B - A)} \cdot 2 \left( \frac{B - A}{2} \right)^3
  = \frac{(B - A)^2}{12}
\end{align*}

\section{Chapter 6 \#2}
We can write
\begin{align*}
  P(X < 3.5\; |\; X \ge 1)
  = \frac{P(1 \le X < 3.5)}{P(X \ge 1)}
  = \frac{\int^{3.5}_1 f(x; 0, 5) dx}{\int^\infty_1 f(x; 0, 5) dx}
  = \frac{\int^{3.5}_1 \frac{1}{5} dx}{\int^5_1 \frac{1}{5} dx}
  = \frac{5}{8}
\end{align*}

\section{Chapter 6 \#10}
Let \(\Phi(z)\) be distribution function of standard normal distribution. Then
we can write
\begin{align*}
  P(\mu - 2\sigma < X < \mu + 2\sigma)
  = \Phi(2) - \Phi(-2)
\end{align*}
According to the table~A.3 in the textbook, we can conclude that the quantity
in question is \(0.9544\).

\section{Chapter 6 \#49}
\subsection{Solution for (a)}
We can write
\begin{align*}
  \mu = \frac{\alpha}{\alpha + \beta} = \frac{1}{1 + 3} = \frac{1}{4}
\end{align*}
For median, we can calculate the integral from 0 to \(t\), where \(0 < t < 1\).
\begin{align*}
  F(t)
  &= \int^t_0 \frac{1}{B(1, 3)} (1 - x)^2 dx
  = \frac{\Gamma(4)}{\Gamma(1) \Gamma(3)} \int^t_0 (1 - x)^2 dx
  = 3 \left[ -\frac{1}{3} (1 - x)^3 \right]^t_0 \\
  &= 1 - (1 - t)^3
\end{align*}
Then, the solution of \(F(t) = 1/2\) is \(t = 1 - \sqrt[3]{1/2}\) and it is the
median.

\subsection{Solution for (b)}
According to the formula of variance of gamma distribution,
\begin{align*}
  \sigma^2
  = \frac{\alpha \beta}{(\alpha + \beta)^2 (\alpha + \beta + 1)}
  = \frac{3}{4^2 \cdot 5}
  = \frac{3}{80}
\end{align*}

\section{Chapter 6 \#53}
\subsection{Solution for (a)}
According to the formula of mean of gamma distribution,
\begin{align*}
  \mu
  = \frac{\alpha}{\alpha + \beta}
  = \frac{5}{5 + 10}
  = \frac{1}{3}
\end{align*}

\subsection{Solution for (b)}
According to the formula of variance of gamma distribution,
\begin{align*}
  \sigma^2
  = \frac{\alpha \beta}{(\alpha + \beta)^2 (\alpha + \beta + 1)}
  = \frac{50}{15^2 \cdot 16}
  = \frac{1}{72}
\end{align*}

\section{Chapter 6 \#58}
\subsection{Solution for (a)}
Let \(N(t)\) be the number of cars passed in first \(t\) minutes. Then \(N(t)\)
follows a \(p(x; 5)\). The probability we are interested in is \(P(N(t) >
10)\).
\begin{align*}
  P(N(t) > 10)
  = 1 - \sum^{10}_{x = 0} P(N(t) = x)
  = 1 - \sum^{10}_{x = 0} \frac{e^{-5} \cdot 5^x}{x!}
  = 0.01370
\end{align*}

\subsection{Solution for (b)}
Let \(X\) be the arrival time of the tenth car. Then the probability we are
interested in is \(P(X > 2)\). Using the relation of exponential distribution
and Poisson distribution,
\begin{align*}
  P(X > 2)
  = P(N(2) < 10)
  = \sum^9_{k = 0} e^{-10} \cdot \frac{10^k}{k!}
  = 0.45793
\end{align*}

\section{Chapter 6 \#59}
\subsection{Solution for (a)}
By definition of Poisson process, the time \(T\) between events follows an
exponential distribution with mean \(1 / 5\). The probability we are interested
in is \(P(T > 1)\).
\begin{align*}
  P(T > 1)
  = \frac{e^{-5} \cdot (5)^0}{0!}
  = e^{-5}
  = 0.00674
\end{align*}

\subsection{Solution for (b)}
As stated earlier, \(T\) follows an exponential distribution with mean \(1 /
5\).

\end{document}
% vim: textwidth=79
