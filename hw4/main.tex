\documentclass{scrartcl}
\usepackage[margin=3cm]{geometry}
\usepackage{amsmath}
\usepackage{amssymb}
\usepackage{amsthm}
\usepackage{blindtext}
\usepackage{datetime}
\usepackage{fontspec}
\usepackage{graphicx}
\usepackage{kotex}
\usepackage{mathrsfs}
\usepackage{mathtools}
\usepackage{pgf,tikz,pgfplots}
\usepackage{float}

\pgfplotsset{compat=1.15}
\usetikzlibrary{arrows}

\newcommand\Overline[2][0.8pt]{%
  \begin{tikzpicture}[baseline=(a.base)]
    \node[inner xsep=0pt,inner ysep=1.5pt] (a) {$#2$};
    \draw[line width= #1] (a.north west) -- (a.north east);
  \end{tikzpicture}
}
\newtheorem{theorem}{Theorem}

\setmainhangulfont{Noto Serif CJK KR}[
  UprightFont=* Light, BoldFont=* Bold,
  Script=Hangul, Language=Korean, AutoFakeSlant,
]
\setsanshangulfont{Noto Sans CJK KR}[
  UprightFont=* DemiLight, BoldFont=* Medium,
  Script=Hangul, Language=Korean
]
\setmathhangulfont{Noto Sans CJK KR}[
  SizeFeatures={
    {Size=-6,  Font=* Medium},
    {Size=6-9, Font=*},
    {Size=9-,  Font=* DemiLight},
  },
  Script=Hangul, Language=Korean
]
\title{MATH230: Homework 4 (due Oct. 9)}
\author{손량(20220323)}
\date{Last compiled on: \today, \currenttime}

\newcommand{\un}[1]{\ensuremath{\ \mathrm{#1}}}
\newcommand{\imag}{\operatorname{Im}}
\newcommand{\real}{\operatorname{Re}}
\newcommand{\Log}{\operatorname{Log}}
\newcommand{\Arg}{\operatorname{Arg}}
\DeclareMathOperator*{\Res}{Res}

\begin{document}
\maketitle

\section{Chapter 5 \#5}
\subsection{Solution for (a)}
Let \(X\) be the number of pipework failure in the chemical plant. Assuming
that the failures of pipework are independent, \(X\) follows a \(b(x; 20,
0.3)\). Then the probability we are looking for is \(P(X \ge 10)\).
\begin{align*}
  P(X \ge 10)
  = 1 - P(X < 10)
  = 1 - \sum^9_{x = 0} b(x; 20, 0.3)
  = 1 - 0.95204
  = 0.04796
\end{align*}

\subsection{Solution for (b)}
The probability we are interested in is \(P(X \le 4)\).
\begin{align*}
  P(X \le 4)
  = \sum^4_{x = 0} b(x; 20, 0.3)
  = 0.23751
\end{align*}

\subsection{Solution for (c)}
The probability of five pipes failing due to operator error is \(P(X = 5)\),
and we can write
\begin{align*}
  P(X = 5)
  = {20 \choose 5} \left( \frac{3}{10} \right)^5
    \left( \frac{7}{10} \right)^{15}
  = 0.17886
\end{align*}
The probability is not that small, so \(p = 0.3\) is reasonable.

\section{Chapter 5 \#22}
Using multinomial distribution, we can write
\begin{align*}
  {8 \choose 5, 2, 1} \left( \frac{1}{2} \right)^5 \left( \frac{1}{4} \right)^2
    \left( \frac{1}{4} \right)^1
  = \frac{21}{256}
\end{align*}

\section{Chapter 5 \#32}
Let \(X\) be the number of missiles in the selection which are not defective.
Then, \(X\) follows a \(h(x; 10, 3, 6)\).

\subsection{Solution for (a)}
The probability we are interested in is \(P(X = 3)\).
\begin{align*}
  P(X = 3)
  = h(3; 10, 3, 6)
  = \frac{1}{6}
\end{align*}

\subsection{Solution for (b)}
The probability we are interested in is \(P(X \ge 1)\).
\begin{align*}
  P(X \ge 1)
  = 1 - P(X = 0)
  = 1 - h(1; 10, 3, 6)
  = \frac{29}{30}
\end{align*}

\section{Chapter 5 \#51}
Let \(X\) be the number of coin tosses required before the person to buy
coffeee is determined. There are \(2^3 = 8\) possible outcomes of three coins,
and two of the outcome is all heads or all tails. Thus, the probability of
determining the buyer in a trial is \(p = 1 - 2/8 = 3/4\). As all coin tosses
are independent, \(X\) follows a \(g(x; p)\). The probability we are interested
in is \(P(X < 3)\).
\begin{align*}
  P(X < 3)
  = \sum^2_{n = 1} pq^{n - 1}
  = \sum^2_{n = 1} \left( \frac{3}{4} \right) \left( \frac{1}{4} \right)^{n - 1}
  = \frac{15}{16}
\end{align*}

\section{Chapter 5 \#57}
Let \(X\) be the number of word-processing errors made in the next page. Then,
as the author makes two word-processing errors per page, \(X\) follows a \(p(x;
2)\).

\subsection{Solution for (a)}
The probability we are interested in is \(P(X \ge 4)\).
\begin{align*}
  P(X \ge 4)
  = \sum^\infty_{x = 4} p(x; 2)
  = \sum^\infty_{x = 4} \frac{e^{-2} 2^x}{x!}
  = e^{-2} \left[ e^2 - \left( 1 + \frac{2}{1} + \frac{2^2}{2!}
    + \frac{2^3}{3!} \right) \right]
  = 0.14288
\end{align*}

\subsection{Solution for (b)}
The probability we are interested in is \(P(X = 0)\).
\begin{align*}
  P(X = 0)
  = p(0; 2)
  = e^{-2}
\end{align*}

\section{Chapter 5 \#75}
Let \(X\) be the first shift that the robot fails. Then \(X\) follows a \(g(x;
0.1)\), and the probability we are interested in is \(P(X \le 6)\).
\begin{align*}
  P(X \le 6)
  = \sum^6_{n = 1} g(x; 0.1)
  = \sum^6_{n = 1} pq^{n - 1}
  = \sum^6_{n = 1}
    \left( \frac{1}{10} \right) \left( \frac{9}{10} \right)^{n - 1}
  = 0.468559
\end{align*}

\end{document}
% vim: textwidth=79
