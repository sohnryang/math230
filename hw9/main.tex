\documentclass{scrartcl}
\usepackage[margin=3cm]{geometry}
\usepackage{amsmath}
\usepackage{amssymb}
\usepackage{amsthm}
\usepackage{blindtext}
\usepackage{datetime}
\usepackage{fontspec}
\usepackage{graphicx}
\usepackage{kotex}
\usepackage{mathrsfs}
\usepackage{mathtools}
\usepackage{pgf,tikz,pgfplots}
\usepackage{float}

\pgfplotsset{compat=1.15}
\usetikzlibrary{arrows}

\newcommand\Overline[2][0.8pt]{%
  \begin{tikzpicture}[baseline=(a.base)]
    \node[inner xsep=0pt,inner ysep=1.5pt] (a) {$#2$};
    \draw[line width= #1] (a.north west) -- (a.north east);
  \end{tikzpicture}
}
\newtheorem{theorem}{Theorem}

\setmainhangulfont{Noto Serif CJK KR}[
  UprightFont=* Light, BoldFont=* Bold,
  Script=Hangul, Language=Korean, AutoFakeSlant,
]
\setsanshangulfont{Noto Sans CJK KR}[
  UprightFont=* DemiLight, BoldFont=* Medium,
  Script=Hangul, Language=Korean
]
\setmathhangulfont{Noto Sans CJK KR}[
  SizeFeatures={
    {Size=-6,  Font=* Medium},
    {Size=6-9, Font=*},
    {Size=9-,  Font=* DemiLight},
  },
  Script=Hangul, Language=Korean
]
\title{MATH230: Homework 9 (due Nov. 6)}
\author{손량(20220323)}
\date{Last compiled on: \today, \currenttime}

\newcommand{\un}[1]{\ensuremath{\ \mathrm{#1}}}
\newcommand{\imag}{\operatorname{Im}}
\newcommand{\real}{\operatorname{Re}}
\newcommand{\Log}{\operatorname{Log}}
\newcommand{\Arg}{\operatorname{Arg}}
\DeclareMathOperator*{\Res}{Res}

\begin{document}
\maketitle

\section{Chapter 8 \#8}
\subsection{Solution for (a)}
Adding up the numbers and dividing them with sample size, 12, we get \(428 / 12
= 35.67\) grams.

\subsection{Solution for (b)}
After sorting the samples in increasing order, the 6th and 7th samples are 31
and 34, respectively. Thus, the median is \((31 + 34) / 2 = 32.5\) grams.

\subsection{Solution for (c)}
Of all the sample, 29 grams appeared the most frequently. Thus, the mode is 29.

\section{Chapter 8 \#23}
\subsection{Solution for (a)}
We can write
\begin{align*}
  \mu
  &= \sum_{x \in \{4, 5, 6, 7\}} x P(X = x)
  = 4 (0.2) + 5 (0.4) + 6 (0.3) + 7 (0.1)
  = 5.3 \\
  \sigma^2
  &= \sum_{x \in \{4, 5, 6, 7\}} (x - \mu)^2 P(X = x)
  = \frac{81}{100}
\end{align*}

\subsection{Solution for (b)}
As sample distribution's mean is same as original distribution's mean, \(\mu_X
= \mu = 5.3\). Also, since \(n = 36\), the variance is \(\sigma^2_X = \sigma^2
/ n = 9 / 400\).

\subsection{Solution for (c)}
We can write
\begin{align*}
  P(\bar{X} \ge 5.5)
  = P(\bar{X} - 5.3 \ge 0.2)
  = P \left( \frac{\bar{X} - 5.3}{0.9 / \sqrt{36}} \ge \frac{4}{3} \right)
\end{align*}
As the sample size is sufficient, by applying the central~limit~theorem, for
\(Z \sim N(0, 1)\) we can write
\begin{align*}
  P \left( \frac{\bar{X} - 5.3}{0.9 / \sqrt{36}} \ge \frac{4}{3} \right)
  \simeq P \left(Z \ge \frac{4}{3} \right)
  = 1 - P \left( X < \frac{4}{3} \right)
  = 0.09121122
\end{align*}
Thus, the probability we are looking for is
\begin{align*}
  P(\bar{X} < 5.5)
  = 1 - P(\bar{X} \ge 5.5)
  = 1 - 0.09121122
  = 0.9087888
\end{align*}

\section{Chapter 8 \#25}
\subsection{Solution for (a)}
Let \(X_1, X_2, X_3, X_4\) be the random samples. They follow normal
distribution of mean \(\mu = 18\), and variance \(\sigma^2 = 3^2\). Let
\(\bar{X}\) be the sample mean. Then, \(\bar{X}\) has mean \(\mu_X = \mu\) and
variance \(\sigma^2_X = \sigma^2 / 4\). Moreover, since \(\bar{X}\) is linear
combination of normally distributed random variables, it is also normally
distributed. Then we can write
\begin{align*}
  P(16 < \bar{X} < 19)
  = \Phi \left( \frac{19 - \mu}{\sigma} \right)
    - \Phi \left( \frac{16 - \mu}{\sigma} \right)
  = 0.6562962
\end{align*}
where \(\Phi\) is probability mass function of standard normal distribution.

\subsection{Solution for (b)}
Let \(\bar{X'}\) be the sample mean of 5 samples. Then, \(P(\bar{X'} > \bar{x})
= 0.2\) holds. As \(\bar{X'}\) is normally distributed with mean \(\mu\) and
variance \(\sigma^2 / 5\), \(\Phi((\bar{x} - \mu) / (\sigma / \sqrt{5})) =
0.8\) should hold. Using a calculator, we can conclude that the value of
\(\bar{x}\) is about 19.12915.

\section{Chapter 8 \#41}
Let \(n = 30\). Then we can write
\begin{align*}
  \frac{(n - 1)S^2}{\sigma^2} \sim \chi^2_{n - 1}
\end{align*}

\subsection{Solution for (a)}
Using a calculator's help,
\begin{align*}
  P(S^2 > 7.338)
  = P \left( \frac{(n - 1) S^2}{\sigma^2} > 42.5604 \right)
  = 0.04996394
\end{align*}

\subsection{Solution for (b)}
Using a calculator's help,
\begin{align*}
  P(2.766 < S^2 < 7.883)
  &= P \left( \frac{(n - 1) S^2}{\sigma^2} < 42.5604 \right)
    - P \left( \frac{(n - 1) S^2}{\sigma^2} < 16.0428 \right) \\
  &= 0.9250851
\end{align*}

\end{document}
% vim: textwidth=79
