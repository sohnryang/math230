\documentclass{scrartcl}
\usepackage[margin=3cm]{geometry}
\usepackage{amsmath}
\usepackage{amssymb}
\usepackage{amsthm}
\usepackage{blindtext}
\usepackage{datetime}
\usepackage{fontspec}
\usepackage{graphicx}
\usepackage{kotex}
\usepackage{mathrsfs}
\usepackage{mathtools}
\usepackage{pgf,tikz,pgfplots}
\usepackage{float}

\pgfplotsset{compat=1.15}
\usetikzlibrary{arrows}

\newcommand\Overline[2][0.8pt]{%
  \begin{tikzpicture}[baseline=(a.base)]
    \node[inner xsep=0pt,inner ysep=1.5pt] (a) {$#2$};
    \draw[line width= #1] (a.north west) -- (a.north east);
  \end{tikzpicture}
}
\newtheorem{theorem}{Theorem}

\setmainhangulfont{Noto Serif CJK KR}[
  UprightFont=* Light, BoldFont=* Bold,
  Script=Hangul, Language=Korean, AutoFakeSlant,
]
\setsanshangulfont{Noto Sans CJK KR}[
  UprightFont=* DemiLight, BoldFont=* Medium,
  Script=Hangul, Language=Korean
]
\setmathhangulfont{Noto Sans CJK KR}[
  SizeFeatures={
    {Size=-6,  Font=* Medium},
    {Size=6-9, Font=*},
    {Size=9-,  Font=* DemiLight},
  },
  Script=Hangul, Language=Korean
]
\title{MATH230: Homework 2 (due Sep. 18)}
\author{손량(20220323)}
\date{Last compiled on: \today, \currenttime}

\newcommand{\un}[1]{\ensuremath{\ \mathrm{#1}}}
\newcommand{\imag}{\operatorname{Im}}
\newcommand{\real}{\operatorname{Re}}
\newcommand{\Log}{\operatorname{Log}}
\newcommand{\Arg}{\operatorname{Arg}}
\DeclareMathOperator*{\Res}{Res}

\begin{document}
\maketitle

\section{Chapter 3 \#3}
We can denote an outcome of a coin toss by length 3 string, \(X_1 X_2 X_3\),
where \(X_1 \in \{H, T\}, X_2 \in \{H, T\}, X_3 \in \{H, T\}\). Then the sample
space \(S\) can be written as follows:
\begin{align*}
  S = \{HHH, HHT, HTH, HTT, THH, THT, TTH, TTT\}
\end{align*}

The values \(w\) of the random variable \(W\) can be assigned as follows:
\begin{figure}[H]
  \centering
  \begin{tabular}{|c|c|}
  \hline
  Sample Space & $w$ \\
  \hline
  \(HHH\)        & 3   \\
  \(HHT\)        & 1   \\
  \(HTH\)        & 1   \\
  \(HTT\)        & -1  \\
  \(THH\)        & 1   \\
  \(THT\)        & -1  \\
  \(TTH\)        & -1  \\
  \(TTT\)        & -3  \\
  \hline
  \end{tabular}
\end{figure}

\section{Chapter 3 \#8}
By the defintion of probability distribution, as \(w\) can have one of \(\{-3,
-1, 1, 3\}\), we can write
\begin{align*}
  f(-3) = \left( \frac{1}{3} \right)^3 = \frac{1}{27}, &\quad
  f(-1) = 3 \left( \frac{2}{3} \right) \left( \frac{1}{3} \right)^2
  = \frac{2}{9} \\
  f(1) = 3 \left( \frac{2}{3} \right)^2 \left( \frac{1}{3} \right)
  = \frac{4}{9}, &\quad
  f(3) = \left( \frac{2}{3} \right)^3 = \frac{8}{27}
\end{align*}
Thus, the probability distribution can be written as
\begin{figure}[H]
  \centering
  \begin{tabular}{|l|llll|}
  \hline
  \(w\)    & -3               & -1              & 1               & 3                \\
  \hline
  \(f(w)\) & \(\frac{1}{27}\) & \(\frac{2}{9}\) & \(\frac{4}{9}\) &
  \(\frac{8}{27}\) \\
  \hline
  \end{tabular}
\end{figure}

\section{Chapter 3 \#9}
\subsection{Solution for (a)}
Using the defintion of probability density function,
\begin{align*}
  P(0 < X < 1) = \int^1_0 f(x) dx = \int^1_0 \frac{2(x + 2)}{5} dx
  = \left[ \frac{x^2}{5} + \frac{4x}{5} \right]^1_0 = 1
\end{align*}
Thus, \(P(0 < X < 1) = 1\).

\subsection{Solution for (b)}
Again using the defintion of probability density function,
\begin{align*}
  P(1/4 < X < 3/4) = \int^{\frac{3}{4}}_{\frac{1}{4}} f(x) dx
  = \int^{\frac{3}{4}}_{\frac{1}{4}} \frac{2(x + 2)}{5} dx
  = \left[ \frac{x^2}{5} + \frac{4x}{5} \right]^{\frac{3}{4}}_{\frac{1}{4}}
  = \frac{1}{2}
\end{align*}

\section{Chapter 3 \#30}
\subsection{Solution for (a)}
By the defintion of probability density function,
\begin{align*}
  \int^\infty_{-\infty} f(x) dx = \int^1_{-1} k(3 - x^2) dx
  = \left[ 3kx - \frac{kx^3}{3} \right]^1_{-1}
  = 6k - \frac{2k}{3} = \frac{16k}{3} = 1
\end{align*}
Thus, \(k\) should be \(3/16\) if \(f\) is a valid probability density function.

\subsection{Solution for (b)}
Using the defintion of probability density function,
\begin{align*}
  P(x < 1/2) = \int^{\frac{1}{2}}_{-\infty} f(x) dx
  = \int^{\frac{1}{2}}_{-1} f(x) dx
  = \left[ 3kx - \frac{kx^3}{3} \right]^{\frac{1}{2}}_{-1} = \frac{99}{128}
\end{align*}

\subsection{Solution for (c)}
The magnitude of error is greater than 0.8 if and only if \(x > 0.8\) or \(x <
-0.8\). Using the defintion of probability density function,
\begin{align*}
  P(|X| > 0.8)
  &= P(X < -0.8) + P(X > 0.8)
  = P(-1 \leq X < -0.8) + P(0.8 < X \leq 1) \\
  &= P(-1 < X < -0.8) + P(0.8 < X < 1)
  = \int^{-0.8}_{-1} f(x) dx + \int^{1}_{0.8} f(x) dx \\
  &= \frac{41}{500} + \frac{41}{500} = \frac{41}{250}
\end{align*}

\section{Chapter 3 \#39}
\subsection{Solution for (a)}
The possible pairs of values of \(X\) and \(Y\), \((x, y)\) are
\begin{align*}
  (x, y) \in \{(0, 1), (1, 0), (0, 2), (1, 1), (2, 0), (1, 2), (2, 1), (3, 0),
  (2, 2), (3, 1)\}
\end{align*}
Let \(f(x, y)\) be the joint probability distribution. Then, \(f(x, y)\) denote
the probability of picking \(x\) oranges, \(y\) apples and \((4 - x - y)\)
bananas. In total, there are \(8 \choose 4\) ways of picking fruits, so we can
write
\begin{align*}
  f(x, y)
  = \frac{{3 \choose x} {2 \choose y} {3 \choose {4 - x - y}}}{{8 \choose 4}}
\end{align*}
for \(x \in \{0, 1, 2, 3\}, y \in \{0, 1, 2\}, 1 \leq x + y \leq 4\).

\subsection{Solution for (b)}
We can write
\begin{align*}
  P[(X, Y) \in A]
  &= P(X + Y \leq 2)
  = f(0, 1) + f(1, 0) + f(0, 2) + f(1, 1) + f(2, 0) \\
  &= \frac{1}{35} + \frac{3}{70} + \frac{3}{70} + \frac{9}{35} + \frac{9}{70}
  = \frac{1}{2}
\end{align*}

\section{Chapter 3 \#40}
\subsection{Solution for (a)}
Let \(g(x)\) be the marginal density of \(X\). By the defintion of marginal
density,
\begin{align*}
  g(x)
  = \int^\infty_{-\infty} f(x, y) dy
  = \int^1_0 f(x, y) dy
  = \int^1_0 \frac{2}{3} (x + 2y) dy
  = \left[ \frac{2xy}{3} + \frac{2y^2}{3} \right]^1_{y = 0}
  = \frac{2x + 2}{3}
\end{align*}
where \(x \in [0, 1]\).

\subsection{Solution for (b)}
Let \(h(y)\) be the marginal density of \(Y\). By the defintion of marginal
density,
\begin{align*}
  h(y)
  = \int^\infty_{-\infty} f(x, y) dx
  = \int^1_0 f(x, y) dx
  = \int^1_0 \frac{2}{3} (x + 2y) dx
  = \left[ \frac{x^2}{3} + \frac{4xy}{3} \right]^1_{x = 0}
  = \frac{4y + 1}{3}
\end{align*}
where \(y \in [0, 1]\).

\subsection{Solution for (c)}
Using the defintion of marginal density, we can write
\begin{align*}
  P(X \leq 1 / 2)
  &= P(X \leq 1 / 2, -\infty < Y < \infty)
  = \int^{\frac{1}{2}}_{-\infty} \int^\infty_{-\infty} f(x, y) dy dx \\
  &= \int^{\frac{1}{2}}_0 g(x) dx = \frac{5}{12}
\end{align*}

\section{Chapter 3 \#47}
\subsection{Solution for (a)}
Let \(g(x), h(y)\) be the marginal deinsities of \(X\) and \(Y\). By the
defintion of marginal density,
\begin{align*}
  g(x)
  &= \int^\infty_{-\infty} f(x, y) dy
  = \int^1_x 2 dy
  = 2(1 - x) \\
  h(y)
  &= \int^\infty_{-\infty} f(x, y) dx
  = \int^y_0 2 dx
  = 2y
\end{align*}
where \(x \in (0, 1)\), \(y \in (0, 1)\).
By plugging \(x = y = 1/2\), we obtain
\begin{align*}
  f(1/2, 1/2) = 2 \not = g(1/2) h(1/2) = 1
\end{align*}
Thus, \(X\) and \(Y\) are dependent.

\subsection{Solution for (b)}
We can write
\begin{align*}
  f(x | y) = \frac{f(x, y)}{h(y)} = \frac{1}{y}
\end{align*}
Then,
\begin{align*}
  P(1/4 < X < 1/2 | Y = 3/4)
  = \int^\frac{1}{2}_\frac{1}{4} f(x | y = 3/4) dx
  = \int^\frac{1}{2}_\frac{1}{4} \frac{4}{3} dx
  = \frac{1}{3}
\end{align*}

\end{document}
% vim: textwidth=79
